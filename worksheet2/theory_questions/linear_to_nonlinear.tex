
%question 1c: How is it possible that we can approximate the non-linear NsE
%             by a linear LB update rule?

\section{Linear LBM to non-linear NSE}

In the LBM we have a linear update rule,

\begin{equation}
    f_i(\vec{x} + \vec{c}_{i}dt, t + dt) = f_i(\vec{x},t) - \frac{1}{\tau} ( f_i(\vec{x},t) - f_{i}^{eq}),
\end{equation}

which gives us the updated distribution of particles with velocity $\vec{c}_i$ 
at position $\vec{x} + \vec{c}_idt$ and time $t+dt$. $i = 0:Q-1$ where $Q$ is 
the size of our velocity space.\\

\subsection{Heuristic argument}

The NSE solves a global system on a macroscale. To get the macroscopic 
velocities from the LBM velocities we need to integrate the velocity 
density distribution over the velocity set. We hence introduce nonlinearity.\\


