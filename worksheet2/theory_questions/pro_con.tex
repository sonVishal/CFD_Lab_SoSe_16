
% question 1a

\section{Pros and cons of LB}

\subsection{Pros}
\begin{itemize}
    \item Explicit update rule, no system of equations to solve.
    \begin{itemize}
        \item Done by doing a trick. Introduce a new distribution function 
              and see that the sum over states give the same global quantities.
    \end{itemize}

    \item Easy to parallelize.
        \begin{itemize}
            \item Each cell only depend locally on the neighbouring cells in
                  the streaming step and only on itself in the collision.
        \end{itemize}

    \item Simulating mesoscale
        \begin{itemize}
            \item Unlike NS, LB looks at distributions of "bunches" of
                  particles. May catch behaviour too fine for NS to detect.
            %\item e.g. fluid running through small cracks. %Find right place to put this
        \end{itemize}
    \item Can handle multiphase flows relatively easy.
        \begin{itemize}
            \item Usually done by modifying the collision operator.
        \end{itemize}
    \item Weakly compressible flows as compared to incompressible in Navier Stokes

\end{itemize}

\subsection{Cons}
\begin{itemize}
    \item Explicit update rule
        \begin{itemize}
            \item Need small timesteps.
            \item Finite propagation speed of information.
        \end{itemize}

    \item Only using a cartesian grid can be limiting.
    \item Memory bound code: You need to store a lot per cell!
	\item Mixed grid sizes are not trivial    

        
        
        

\end{itemize}
