\documentclass[a4paper]{article}
\usepackage{amsmath}
\usepackage{float}
\usepackage{amssymb}
\usepackage{todonotes}
\usepackage{hyperref}
\usepackage{listings}
\usepackage{xcolor}
\usepackage{cancel}

\newcommand\norm[1]{\left\lVert#1\right\rVert}		%norm
\newcommand{\td}[1]{\todo[inline]{#1}}  			%inline todo
\newcommand{\qm}[1]{``#1"}							%quotation marks
\newcommand{\ra}{\rightarrow}						%right array
\newcommand{\NP}{\noindent\rule{\textwidth}{1pt}\newline} 	%separator line
\newcommand{\abs}[1]{\left\vert#1\right\vert} 			%absolute values
\newcommand{\dotp}[2]{\langle#1,#2\rangle} 			%dotproduct
\newcommand{\seq}[3]{(#3_{#1}, \ldots, #3_{#2})}


\begin{document}

\part{Worksheet 3}

\section{Backlog}
\begin{enumerate}
\item \textbf{Tool to create scenarios (domain + obstacles)}
\begin{enumerate}
\item \td{Group disussion:}
what \qm{tool} are we using to create domains? (consider: should be easy to \qm{integrate} and flexible enough to build \qm{complex} scenarios)
\item[] \textbf{Programmatically}... Matlab/Python - just create a boolean matrix, and tell which "pixels" are obstacles, matrix can also be plotted easily, etc. There are also scaling functions to make it larger, etc.
\item[] \textbf{Graphically}... \qm{Paint}-like drawing and then reading via a script?
\item \td{Group discussion:}
Do we use \qm{read\_pgm}? (see page 12 (3) \url{https://en.wikipedia.org/wiki/Netpbm_format})
\end{enumerate}

\item \textbf{Adapt parameter file, read parameter and check for correctness}
\begin{enumerate}
\item Variable length in all three directions (x\_length, y\_length, z\_length); adapt signature of functions; see page 12 (1)
\item substitute \qm{wallVelocity} by an array of parameter, for inflow condition; see page 12 (2)
\end{enumerate}

\item \textbf{Adapt initializeFields according to the current problem we are simulating; see page 12 (3)}
\begin{enumerate}
\item extent signature of initializeFields with "*char problem", which contains the name of scenario
\item set flagField according to the current geometry (created in step 1)
\item using of \qm{read\_pmg} \todo{Decide !!} can be used to initialize flagField 
\item Check for correctness, is there is any illegal boundary (too thin)?
\end{enumerate}

\item  \textbf{Write treatBoundary again}
\begin{enumerate}
\item implement other boundary conditions (Free-Slip, Inflow, Outflow); see page 12 (2)
\item rewrite iteration such that arbitrary geometries are handled
\item try to improve performance (that can, for example, involve additional steps in the initialization of the simulation)
\end{enumerate}

\item \textbf{Write WriteVTKOutput again}
\begin{enumerate}
	\item Currently it is restricted to the previous worksheet
	\item One possible way would be to use \qm{read\_pgm}\todo{Decide!!} and store the column and line number to help in creating the coordinates corresponding to different types of cells. Then iterate over the flag field and print out only the fluid cells
\end{enumerate}

\item \textbf{Scenarios}
\begin{enumerate}
\item Create scenarios given in worksheet (page 13)
\item Save results and paraview visualization files in a special folder, (Note: page 13, \qm{Please bring your computer[...] to speed up the review process})
\item Create some own scenarios/tests
\item \qm{Use the features provided by ParaView excessively to analyze the flows.} $ \ra $ generate outputs and save them for discussion/understanding/oral exam
\end{enumerate}

\item \textbf{Tests for plausibility}
\begin{enumerate}
\item Find given given examples and see if our results match
\item Test small trivial examples
\item Test \qm{free-slip} with half the domain against the normal simulation for example of \qm{Plane shear force} (see section 2.4(b) page 12)
\end{enumerate}

\end{enumerate} 


\section{Consider from worksheet!}
\begin{itemize}
\item Changes of parameters or boundary conditions should no longer require modifications in the source code, nor recompilation of the program.
\item Demonstrate, that the program works properly by providing solutions for the examples shown in the last section.
\item Implement the possibility to perform computations of the problems that are defined by setting the boundary pressure. Check your program with the plane shear flow and the flow over a step.
\item With the help of ParaView, visualize the Karman vortex street and the flow over a step. Use the features provided by ParaView excessively to analyze the flow
\end{itemize}


\end{document}